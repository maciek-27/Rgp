\documentclass[11pt,a4paper]{article}
\RequirePackage[utf8]{inputenc}
\RequirePackage{times}
\RequirePackage{indentfirst}
\usepackage{listings}
\RequirePackage[T1]{fontenc}
\newcommand{\Class}[1]{\texttt{\textbf{#1}}}
\date{January 29, 2008\\last updated: February 7, 2008}
\author{Maciej Kamiński}
\title{RexIO library 1.1 design requirements}
\begin{document}
\maketitle

\begin{abstract}
  Following document describes design requirements for 1.1 version of
  RexIO library specification. It covers basic functionality, that has
  to be implemented, design patterns that must be obeyed and other
  project goals, that need do be achieved. 
\end{abstract}

\section{RexIO Platform}

Current (1.0) version of RexIO library covers two basic functional
modules: first implemented as librexio, covering basic screen operations
and second implemented as librexiotk, encompassing basic UI
construction toolkit. In 1.1 third module will be included, covering
networking procedures. This will be classes representing running
server, active client connection etc.

These three modules together will form comprehensive application
framework sufficient to easily create online, multi-user software,
that provides  customizability,  performance and 
maintainability. Please note, that this is very early stage of design,
and everything may change before we start implementing it.

\section{RexIO 1.1 networking module (\texttt{RexIO::Net})}

The networking module will (as said above) contain set of classes
implementing basic features of typical network application. Namespace
for this module will be 

\begin{itemize}
\item \Class{Server} class --- implementing listening
  socket. Creating \Class{ClientConnection} object for each incoming
  connection. May easily be started to listen, and stopped (when
  associated \Class{ConnectionPool} is empty).
\item \Class{ConnectionPool} implementing set of active client connections
  associated with running Server.
\item \Class{ClientConnection} object is created for every incoming
  connection by Server. It contains pair of standard C++ streams to
  communicate with client. New thread is created by RexIO for each
  ClientConnection.
\end{itemize}

\section{RexIO 1.1 screen module (\texttt{RexIO::Scr})}

Core library module will also encounter major improvements

\begin{itemize}
\item \underline{Support for mouse operations.} These improvements may
  require changes in all descendents of \Class{\_\_ScreenConnection}
  class, which, \textit{nb.} will be renamed to
  \Class{ScreenConnection} as well as adding new functions to
  \Class{Connection} object
\item Full level 1 of Unicode implementation, including RTL and LTR
  text directions.
\item Optimize screen operations in terms of both: speed and transfer.
\item Introduce support for \texttt{SIGWINCH}
\item Recognize full PC keyboard, including recognition of as many as
  possible keys with Ctrl/Alt/Shift pressed (yes, this includes
  processing \texttt{UNIX} signals for local connections).
\item imtroduce Extended/Buffered subwindow types
\end{itemize}

\section{RexIO 1.1 UI toolkit module (\texttt{RexIO::Scr::Tk})}

\begin{itemize}
\item Make use of mouse support
\item Introduce new widgets crucial for real-life application, that
  including (but not limited to)
  \begin{itemize}
  \item TextEdit
  \item ListBox
  \item Frameset 
  \item Table (as for spreadsheet programs)
  \item A window, that may be scrolled
  \item TextMap (or TextCanvas) - component for buffered, incrementally 
	changed ascii-art graphics, rogue-like game maps etc.
  \end{itemize}
\item Introduce better support for L10n/i18n
\item Improve stylesheet support
\end{itemize}

\section{Summary}
1.1 will be a comprehensive platform for real-life multi-user applications,
including important features, which, due to lack of time, are absent
in 1.0. Enhanced interface (including networking module) will enable
easy transition to portability to non-UNIX operating systems.

\section*{Appendix A: proposal for programming interface %
of Screen for accessing new types of subscreens in Doxygen/C++ format}
\lstset{commentstyle=\textit,stepnumber=1000 ,keywordstyle=\bfseries,
    showstringspaces=false}
\lstset{numbers=none,numbersep=0.4cm,numberstyle=\tt,tabsize=4,
    basicstyle=\tt\scriptsize,texcl=false}
\lstset{language=C++,breaklines=true}
\begin{lstlisting}
class Scr::Screen
    {
		...
		...existing interface...
		...
		public:
		...
		//new functions
		/*!
		  \param _y_offset vertical offset from top edge of this
		  screen to top edge of new SubScreen.
		  \param _x_offset horizontal offser
		  \param _h height
		  \param _w with
		  \return  pointer to new BufferedSubScreen
		  (programmer will have to free it's
		  resources to prevent memory leak and other errors).
		  BufferedSubScreen differs from basic SubScreen in many ways,
		  in particular it has it's own buffer and therefore it's
		  operations does not affect parent screen unless it's
		  refreshed. 
	  
		  \exception Scr::Screen::SubscreenOutOfRange
		  is thrown when too big subscreen requested or inapropriate
		  position specified
		*/
		virtual Screen *
		CreateBufferedSubScreen(Uint _y_offset,
								Uint _x_offset, Uint _h,
								Uint _w)throw(SubscreenOutOfRange);
	
		/*!
		  \param _y_offset vertical offset from top edge of this
		  screen to top edge of new SubScreen.
		  \param _x_offset horizontal offser
		  \param _h height
		  \param _w with
		  \param ext_left extended range-checking limit
		  \param ext_right extended range-checking limit
		  \param ext_top extended range-checking limit
		  \param ext_bottom extended range-checking limit
		  \return  pointer to new SmartSubScreen
		  (programmer will have to free it's
		  resources to prevent memory leak and other errors).
		  Key differnces between SmartSubScreen and basic SubScreen is,
		  that it:
		  - shifts contents up by ext_top and left by ext_left
		  - allows writing space of (ext_left + _w + ext_right)
		  horizontal and (ext_top + _h + ext_bottom) vertical (so it's
		  internal size is bigger than displayed size). It crops
		  everything, that does not fit in this internal size, but
		  fits in external. It throws exceptions connected w/ range
		  only if internal range limit is broken.	  
	  
		  \exception Scr::Screen::SubscreenOutOfRange
		  is thrown when too big subscreen requested or inapropriate
		  position specified	  
		*/
		virtual Screen *
		CreateExtendedSubScreen(Uint _y_offset,
								Uint _x_offset, Uint _h,
								Uint _w,Uint ext_left, Uint ext_right, 
								Uint ext_top, Uint ext_bottom)
			throw(SubscreenOutOfRange);
	
		/*!
		  \param _y_offset vertical offset from top edge of this
		  screen to top edge of new SubScreen.
		  \param _x_offset horizontal offser
		  \param _h height
		  \param _w with
		  \param ext_left left shift_and_crop of content
		  \param ext_top top shift_and_crop of content
		  \return  pointer to new SmartSubScreen
		  (programmer will have to free it's
		  resources to prevent memory leak and other errors).
		  Key differnces between SmartSubScreen and basic SubScreen is,
		  that it:
		  - shifts contents up by ext_top and left by ext_left
		  - turns of range checking
		  - crops out-of-range content
	  
		  \exception Scr::Screen::SubscreenOutOfRange
		  is thrown when too big subscreen requested or inapropriate
		  position specified

		  \note no range checking means, that programmer gets no
		  information on result of his actions
		*/

		virtual Screen *
		CreateUnlimitedExtendedSubScreen(Uint _y_offset,
										 Uint _x_offset, Uint _h,
										 Uint _w,Uint ext_left,
										 Uint ext_top
			)throw(SubscreenOutOfRange);
		//end of new functions
		...
		...existing interface...
		...
    } 
\end{lstlisting}

\end{document}