\subsection {Typical program layout - message processing pipeline}
Typical software program consists of many functional components, that
is i.e. some file-reading component, networking subsystem, many other
items... and last, but not least, user interface. In object oriented
languages, such as C++, many of them are implemented as classes, so no
surprise, RexIO is designed to let programmers take advantage of OOP
in user interface design. 

As a result each UI element is implemented as OBJECT, and each
customized element is represented by custom class. Therefore designing
user interface is explicitly designing class implementing specific
interface. Typical UI design is as follows

{\footnotesize\begin{verbatim}
Specific window type ``MyWindow'' is a RootWindow
  When it is resized, 
    it does some action

  When key is pressed, 
    it stores it's code if it is a letter
    program quits if it is Esc

  When it has to display its contents
    it displays stored value
\end{verbatim}}

Design above may be directly converted to C++ code as follows

\begin{lstlisting}
class MyWindow: public RootWindow
{
private:
	int code;
public:
	MyWindow()
		:RootWindow(std::cin,std::cout)
	{
		code=0;
	}


	void OnResize()
	{
		;//do something
	}

	void  OnKeyDown(Key key)
	{
		if (key == Key::Escape)
			Exit(0);
		else if (key.IsABasicKey())
			code=key;
	}

	void OnRedraw(Scr::Screen &screen)
	{
		screen << Clear << GotoYX(2,2) << code << Refresh;
	}

	~MyWindow(){;}
};//MyWindow
\end{lstlisting}

\index{class}
Code above depicts almost complete RexIO application. Full program is as follows:
\fullcode
\begin{lstlisting}
#include<rexio/tk/toolkit.h++>
#include<iostream>
using namespace std;
using namespace Scr;
using namespace Scr::Tk;
using namespace Scr::Control;

class MyWindow: public RootWindow
{
private:
	int code;
public:
	MyWindow()throw()  		// empty specification of
							// throw() means, that function
        					// is not allowed to throw
							// any exceptions.
		:RootWindow(cin,cout)
	{
		code=0;
	}

	void OnResize()throw()
	{
		;//do something
		RootWindow::OnResize();
	}

	void  OnKeyDown(Key key)throw()
	{
		if (key == Key::Escape)
			Exit(0);
		else if (key.IsABasicKey())
			code=key;
		RootWindow::OnKeyDown(key);
	}

	void OnRedraw(Scr::Screen &screen)throw()
	{
		try
		{
			screen << Clear << GotoYX(2,2) << code << Refresh;
		}
		catch (...)
		{
			Exit(1);
		}
	}
	
	~MyWindow()throw(){;}
};//MyWindow

int main (Uint argc,char ** argv)
{
	RootWindow * app = new MyWindow;	
	int result = app->Start(argc, argv);	
	delete app;
	return result;
}
/*end of main function of program*/ 
\end{lstlisting}

As you can see, line 
\vspace{-10pt}\begin{verbatim}
#include<rexio/tk/toolkit.h++>
\end{verbatim}
\vspace{-10pt}includes most general library header file (please note, that this file
includes virtually ,,everything'' - there are also files declaring specific
classes)

\index{namespace}
lines 
\vspace{-10pt}\begin{verbatim}
using namespace std;
using namespace Scr;
using namespace Scr::Tk;
using namespace Scr::Control;
\end{verbatim}
\vspace{-10pt}aren't necessary to make code working, however they allow to simplify
many statements.

Keyword
\vspace{-10pt}\begin{verbatim}
throw
\end{verbatim}
\vspace{-10pt}is used in whole library to specify allowed exception sets, and
therefore enable controlling exception flow. 

Sometimes, when redefining default behavior of windows (especially
RootWindow) it is recommended to call default function after (sometimes
before) custom processing:\vspace{-10pt}
\begin{verbatim}
RootWindow::OnKeyDown(key);
\end{verbatim}

\vspace{\fill}
\pagebreak
\subsection {Basic character output}
\index{output}

\index{stream}
In previous section we have discussed basic printing text
using following sequence
\begin{lstlisting}
screen << Clear << GotoYX(2,2) << "Hello World" << Refresh;//>>
\end{lstlisting}

The same effect may be obtained using plain virtual function calls
\begin{lstlisting}
screen.Clear();
screen.GotoYX(2,2);
screen.AddText("Hello World");
screen.Refresh();
\end{lstlisting}


Please note, that there are multiple (to be precise: 6) variants of
AddText: let us consider two of them
\begin{lstlisting}
virtual void AddText(const char * text)   
virtual void AddText(const wchar_t * text) 
\end{lstlisting}

One of them accept C-style string with one-byte-per-character, and
second accepts wchar\_t (for Linux 4 byte, for Windows 2 byte). 

\index{Jožin z bažin}
The second may be used to print text with diacritics. i.e. to print
,,Jožin z bažin'' you have to type following code:

\noindent{\tt\scriptsize screen.AddText(L"Jožin z bažin");}


\index{UCS}
However as many real software solutions depend on UTF-8 encoding,
specific functionality \textbf{is provided out of the box} 
\begin{lstlisting}
screen.AddText("Jo\xC5\xBEin z ba\xC5\xBEin");
\end{lstlisting}
does exactly, what you may expect
\index{UTF-8}
\index{color}
If you want to emphasize "bažin", you may use following code to add colors:
\begin{lstlisting}
screen << "Jo\xC5\xBEin z " << Fg::Bright << Fg::Red << "ba\xC5\xBEin";//>>
\end{lstlisting}


\index{stream}
SetFgStyle and SetFgColor functions may be called instead of using
this iostream-like syntax. Maybe you won't gain any bigger performance
using these functions, but certainly you may improve control of
overall layout. Also there are functions like Screen::HorizontalLine
simplifying for example box drawing. 

\important
Each of these  functions provides range checking, and throws specific
exception when range is violated. It is recommended to use try-catch
statements to detect such problems and provide rock-solid programs that
virtually never fail. 

%w razie wolnego czasu coś o CJK napisać trzeba

\vspace{\fill}
\pagebreak


\subsection {Component-based hierarchical layout}


\index{component}

To improve your understanding of hierarchical layout (basics of
Scr::Tk::Widget usage) please consider this piece of code as example.

\fullcode
\lstinputlisting{../../test/3/src/test.c++}